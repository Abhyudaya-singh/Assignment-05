\documentclass[12pt]{article}
\usepackage[english]{babel}
\usepackage{natbib}
\usepackage{url}
\usepackage[utf8x]{inputenc}
\usepackage{amsmath}
\usepackage{graphicx}
\graphicspath{{images/}}
\usepackage{parskip}
\usepackage{fancyhdr}
\usepackage{vmargin}
\setmarginsrb{3 cm}{2.5 cm}{3 cm}{2.5 cm}{1 cm}{1.5 cm}{1 cm}{1.5 cm}

\title{Emerging Technologies In Healthcare}					
\author{21111004}								
\date{25 FEB 2022}								
\makeatletter
\let\thetitle\@title
\let\theauthor\@author
\let\thedate\@date
\makeatother

\pagestyle{fancy}
\fancyhf{}
\rhead{\theauthor}
\lhead{\thetitle}
\cfoot{\thepage}

\begin{document}
\begin{titlepage}
	\centering
    \includegraphics[scale = 0.20]{logo.jpg}\\[1.0 cm]	
    \textsc{\LARGE National Institute Of Technology \newline\\\\ RAIPUR}\\[2.0 CM]
    
	\textsc{\Large ASSIGNMENT 05}\\[0.5 cm]				% Course Code
	\rule{\linewidth}{0.4 mm} \\[0.4 cm]
	{ \huge \bfseries \thetitle}\\
	\rule{\linewidth}{0.4 mm} \\[1.5 cm]
	
	\begin{minipage}{0.6\textwidth}
		\begin{flushleft} \large
			\emph{Submitted To:}\\
			Saurabh Gupta\\
            Department Of Basic Biomedical Engineering\\
			\end{flushleft}
			\end{minipage}~
			\begin{minipage}{0.4\textwidth}
            
			\begin{flushright} \large
			\emph{Submitted By :}\\
			Abhyudaya Kumar Singh\\
            21111004\\
		\end{flushright}
        
	\end{minipage}\\[2 cm]
\end{titlepage}

\tableofcontents
\pagebreak

\section{Emerging Technologies In Healthcare}
It is hard to imagine a world now without sophisticated medical devices, efficient electronic data transfer of real-time health information and high-tech operating rooms – health IT solutions that are improving outcomes and transforming healthcare industry.
Medical technology has come a long way since the invention of eyeglasses and the stethoscope. The broader availability of mobile internet, the expansion of a more affluent middle class, and an aging global population are all driving change in the healthcare industry, and the associated technology is changing faster than ever before.


\subsection{Internet of Medical Things}
Today's internet-connected devices in any industry are specially designed to enhance efficiencies and lower costs. To drive better outcomes in the healthcare sector, the Internet of Medical things is made for computing power and wireless capabilities where organizations can take advantage of the potential of the IoMT technologies.
IoMT applications play a crucial role in tracking and controlling chronic illnesses for clinics and patients, and they're poised to become the future of healthcare.
In short, the Internet of Medical Things (IoMT) is a combined infrastructure of services to rural and urban healthcare systems. And for patients who are unable to get their healthcare provider, IoMT is the solution.
With the help of IoMT devices, one can monitor patients' health remotely, also known as telemedicine.

\subsection{Artificial Intelligence in Healthcare}
Artificial intelligence (AI) has the potential to change the industry's future. It transforms how healthcare is delivered traditionally and upgraded with new aspects and trends for 2022.
A combined study with the European Union's EIT Health examines how AI is helpful can assist in enhancement to healthcare, patient experience, and access to other healthcare tools and services.
In 2022, hospitals will dig more into medical tools. With AI, you can see more effectiveness of infection prevention and control (IP&C) programs powered by artificial intelligence (AI) to monitor patients in real-time. With the help of Artificial Intelligence, care providers will get fast infection risk designation and early clinical intervention.
\subsection{Digital Therapeutics}
Digital Therapeutics (DTx) is a digital health category as the Alliance of Digital Therapeutics describes. It is used as the effect that delivers evidence-based therapeutic interventions. It is for patients that help to drive high-quality software programs to control, supervise, or feast on a medical disorder or disease.
Digital Therapeutics is a patient-centred movement in the healthcare industry that provides evidence-based care through software development services. It can also evolve a full substitute or the existing disease treatment.
The healthcare technology, especially for DTx, is developed for clinicians and patients and contains clinical assessment, patient-reported outcomes, and clinician monitoring dashboards.
Predictive analytics strives to warn clinicians and guardians of the likelihood of events and outcomes before they occur, allowing them to control as much as cure health issues.
\subsection{Predictive Analytics in Healthcare}
As a known software development company, it is driven by IoT and Artificial Intelligence (AI) algorithms that can be fed with recorded and real-time data for insightful predictions and solutions.
To help clinical decision-making, some predictive algorithms support individual patients and brief interventions on a population level.
\subsection{Big Data in Healthcare}
Big data is a massive door opening for the healthcare industry where multiple sources include medical records of patients, hospital records, effects of medical examinations, and devices that are a part of the internet of things.
Biomedical research will help you create an important part of big data that applies to public healthcare. This insightful data need structure and high management to analyze and structure data for the healthcare industry.
Electronic Health Records (EHRs) allow faster data rescue, promote key healthcare quality indicators to the organizations, and improve public health surveillance by immediately reporting disease outbreaks.
 \subsection{Cloud Computing}
By delivering on-demand healthcare storage, cloud computing in healthcare lowers capital and operational expenses. It increases the efficiency of the industry while cutting costs.

Cloud computing makes it easy to share medical records, automate backend operations, and even create and maintain telehealth apps.
Cloud computing enables high-powered analytics for more customized patient care plans using EMRs stored in the cloud.


\end{document}

